\documentclass[16pt]{beamer}
\usepackage[utf8]{inputenc}
\usepackage[T1]{fontenc}
\usepackage{graphicx}
\usepackage[polish]{babel}
\usepackage{url}
\input{pygments}

\usetheme{Pittsburgh}
\usenavigationsymbolstemplate{} % turn off navigation icons
\setbeamercovered{transparent}

\author{Silesian Ruby Users Group\\\footnotesize{Jakub Kuźma}}
\title{Lisp - Introduction}

\begin{document}

\frame{\titlepage}

\begin{frame}
  \frametitle{WTF?}
  \begin{center}
    Lisp?!?
  \end{center}
\end{frame}

\begin{frame}
  \frametitle{John McCarthy\\September 4, 1927 -- October 24, 2011}
  \begin{figure}
    \includegraphics[width=0.8\linewidth]{mccarthy.jpg}
  \end{figure}
\end{frame}

\begin{frame}
  \frametitle{Atoms}
  \begin{block}{}
    Atoms are strings of letters and digits and other characters not
    otherwise used in Lisp.
  \end{block}
  \begin{itemize}
  \item 0, 42, 3.14
  \item ``hello, world!''
  \item foo, car, +
  \item t
  \end{itemize}
\end{frame}

\begin{frame}
  \frametitle{Lists}
  \begin{block}{}
    A list consist of a left parenthesis followed by zero or more
    atoms or lists separated by spaces and ending with a right
    parenthesis.
  \end{block}
  \begin{itemize}
  \item ()
  \item (foo)
  \item (1 + 2)
  \item (foo (bar (baz)))
  \end{itemize}
\end{frame}

\begin{frame}
  \frametitle{Symbolic Expressions}
  \begin{block}{}
    Not all s-expressions are valid Lisp programs.
  \end{block}
\end{frame}

\begin{frame}
  \frametitle{Primitives}
  \begin{itemize}
  \item (quote $e$)
  \item (car $e$)
  \item (cdr $e$)
  \item (cons $e_1$ $e_2$)
  \item (equal $e_1$ $e_2$)
  \item (atom $e$)
  \item (cond ($p_1$ $e_1$) ... ($p_n$ $e_n$))
  \item An atom $v$, regarded as a variable, may have a value.
  \item ((lambda ($v_1$ ... $v_n$) $e$) $e_1$ ... $e_n$)
  \item ((label $f$ (lambda ($v_1$ ... $v_n$) $e$)) $e_1$ ... $e_n$)
  \end{itemize}
\end{frame}

\begin{frame}
  \frametitle{Calling functions}
  \begin{block}{}
    The most common way of invoking a function in Lisp is by evaluating a list.
  \end{block}
  \begin{itemize}
  \item (+ 2 2)
  \item (* (+ 1 3) 5)
  \item (concat ``a'' ``b'')
  \end{itemize}
\end{frame}

\end{document}
